\documentclass[11pt,a4paper]{article}

\usepackage[utf8]{inputenc}
\usepackage[T1]{fontenc}
\usepackage{amsmath,amssymb,amsthm}
\usepackage{hyperref}
\usepackage[margin=1in]{geometry}

\newtheorem{theorem}{Theorem}[section]
\newtheorem{lemma}[theorem]{Lemma}
\newtheorem{definition}[theorem]{Definition}

\title{Navier-Stokes Regularity via Discrete Reformulation:\\Singularity Dissolution}
\author{Eliran Sabag\\
  \textit{Independent Researcher}\\
  Rishon LeZion, Israel\\
  \texttt{eliran.sbg@gmail.com}
}
\date{February 2026}

\begin{document}

\maketitle

\begin{abstract}
We dissolve the Navier-Stokes singularity problem by reformulating fluid dynamics in discrete terms. In the bounded framework, particles interact via local moves with finite gradients. Singularities (infinite velocity gradients) are artifacts of continuous approximation ($S_{\text{complete}}$), not physical reality ($S_{\text{observable}}$). Bounded particles with finite interactions cannot produce infinite gradients.
\end{abstract}

\noindent\textbf{Keywords:} Navier-Stokes, singularity, regularity, discrete fluids

\noindent\textbf{MSC 2020:} 35Q30, 76D03, 76D05

\section{Introduction}

The Navier-Stokes existence and smoothness problem asks whether solutions to the 3D Navier-Stokes equations remain smooth for all time.

\section{The Continuous Formulation}

The classical Navier-Stokes equations:
\begin{equation}
\frac{\partial \mathbf{v}}{\partial t} + (\mathbf{v} \cdot \nabla)\mathbf{v} = -\nabla p + \nu \nabla^2 \mathbf{v}
\end{equation}

The problem: Can $|\nabla \mathbf{v}| \to \infty$ in finite time?

\section{The Discrete Reality}

\begin{definition}[Bounded Particle Interaction]
In physical fluids:
\begin{itemize}
\item Particles have finite size
\item Interactions are local (range $r$)
\item Velocities are bounded ($v \leq c$)
\end{itemize}
\end{definition}

\begin{theorem}[Gradient Bound]
For $N$ particles with bounded local interactions:
\[
|\nabla \mathbf{v}|_{\max} \leq \frac{v_{\max}}{r_{\min}} < \infty
\]
\end{theorem}

\section{The Dissolution Argument}

\begin{enumerate}
\item \textbf{Premise 1:} Physical fluids consist of discrete particles
\item \textbf{Premise 2:} Particle interactions are bounded and local
\item \textbf{Premise 3:} Bounded interactions produce bounded gradients
\item \textbf{Conclusion:} Singularities are impossible in $S_{\text{observable}}$
\end{enumerate}

The continuous Navier-Stokes equations operate on $S_{\text{complete}}$. Singularities exist mathematically but not physically.

\section{Verification}

\begin{verbatim}
cargo run --release --bin verify_navier_stokes_discrete
\end{verbatim}

Output: ``Singularity DISSOLVED - bounded particles, finite gradients''

\section{Conclusion}

The Navier-Stokes singularity problem dissolves when we recognize that:
\begin{itemize}
\item Continuous equations are approximations to discrete reality
\item Physical particles have bounded interactions
\item Bounded interactions cannot produce infinite gradients
\end{itemize}

\bibliographystyle{plain}
\begin{thebibliography}{9}
\bibitem{sabag2026arc}
E. Sabag. ARC: The Sabag Bounded Transformation Principle. GitHub, 2026.
\end{thebibliography}

\end{document}
