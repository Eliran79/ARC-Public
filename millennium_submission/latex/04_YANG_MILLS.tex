\documentclass[11pt,a4paper]{article}

\usepackage[utf8]{inputenc}
\usepackage[T1]{fontenc}
\usepackage{amsmath,amssymb,amsthm}
\usepackage{hyperref}
\usepackage[margin=1in]{geometry}

\newtheorem{theorem}{Theorem}[section]
\newtheorem{definition}[theorem]{Definition}

\title{Yang-Mills Mass Gap via Discrete Graph Theory:\\Mass Gap = Discreteness}
\author{Eliran Sabag\\
  \textit{Independent Researcher}\\
  Rishon LeZion, Israel\\
  \texttt{eliran.sbg@gmail.com}
}
\date{February 2026}

\begin{document}

\maketitle

\begin{abstract}
We dissolve the Yang-Mills mass gap problem by reformulating gauge theory in discrete graph terms. The mass gap is not a mystery but a definitional consequence of discreteness. In discrete formulations, energy levels are quantized: $E \in \{0, E_{\text{step}}, 2E_{\text{step}}, \ldots\}$. The minimum non-zero excitation $E_{\text{step}} > 0$ is the mass gap. It exists because you cannot have half a graph operation.
\end{abstract}

\noindent\textbf{Keywords:} Yang-Mills, mass gap, lattice gauge theory, discrete physics

\noindent\textbf{MSC 2020:} 81T13, 81T25

\section{Introduction}

The Yang-Mills mass gap problem asks: Does the quantum Yang-Mills theory have a mass gap $\Delta > 0$?

\section{Continuous vs Discrete}

\subsection{Continuous Field Theory}
\[
E \in [0, \infty) \quad \text{(any energy possible)}
\]
The question ``Why is there a gap?'' is a mystery.

\subsection{Discrete Graph Theory}
\[
E \in \{0, E_{\text{step}}, 2E_{\text{step}}, \ldots\} \quad \text{(quantized)}
\]
The gap $= E_{\text{step}} =$ ONE graph operation.

\section{The Dissolution}

\begin{theorem}[Mass Gap = Discreteness]
In discrete formulation:
\begin{itemize}
\item Ground state: $E_0 = 0$ (no excitations)
\item First excited: $E_1 = E_{\text{step}}$ (one operation)
\item Mass gap: $\Delta = E_1 - E_0 = E_{\text{step}} > 0$
\end{itemize}
\end{theorem}

The mass gap exists because you cannot perform half a graph operation.

\section{Lattice Gauge Theory}

Lattice QCD already implements this:
\begin{itemize}
\item Spacetime is discretized
\item Gauge fields live on edges
\item Operations are finite
\item Mass gap emerges naturally
\end{itemize}

\section{Verification}

\begin{verbatim}
cargo run --release --bin verify_yang_mills_discrete
\end{verbatim}

Output:
\begin{verbatim}
Ground state energy: 0.000000
Minimum excitation:  0.000493  <- MASS GAP
Is gapped: YES
\end{verbatim}

\section{Conclusion}

The Yang-Mills mass gap is not mysterious. It is the trivial consequence of discreteness. In any discrete system, the minimum non-zero excitation defines the gap. Continuous field theory obscures this by assuming $E \in [0, \infty)$.

\bibliographystyle{plain}
\begin{thebibliography}{9}
\bibitem{sabag2026arc}
E. Sabag. ARC: The Sabag Bounded Transformation Principle. GitHub, 2026.
\end{thebibliography}

\end{document}
