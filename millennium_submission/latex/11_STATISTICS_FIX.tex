\documentclass[11pt]{article}
\usepackage{amsmath,amssymb,amsthm}
\usepackage[margin=1in]{geometry}
\usepackage{booktabs}

\newtheorem{theorem}{Theorem}
\newtheorem{definition}{Definition}
\newtheorem{corollary}{Corollary}

\title{How to Fix Statistics\\
\large The Two Randomness Theorem and 40 Distributions}
\author{Eliran Sabag}
\date{February 5, 2026}

\begin{document}
\maketitle

\begin{abstract}
We present a systematic correction of probability theory based on the Two Randomness Theorem. Of 40 standard statistical distributions, 26 are ``ethers''---mathematical artifacts arising from unbounded continuous assumptions that don't correspond to physical reality. We provide bounded corrections and an empirical compression test to distinguish physics-level randomness (compressible, 15-92\%) from bit-level randomness (incompressible, 0\%).
\end{abstract}

\section{The Two Types of Randomness}

\begin{definition}[Bit-Level Randomness]
A sequence $x$ is bit-level random if its Kolmogorov complexity satisfies:
$$K(x) \geq |x| - O(1)$$
Such sequences are \textbf{incompressible} with compression ratio $\approx 0\%$.
\end{definition}

\begin{definition}[Physics-Level Randomness]
A sequence $x$ is physics-level random if:
$$K(x) < |x| - \omega(1)$$
Such sequences are \textbf{compressible} with ratio 15-92\%.
\end{definition}

\begin{theorem}[Two Randomness Theorem]
Physical processes produce physics-level randomness (compressible). Only cryptographic sources produce bit-level randomness (incompressible).
\end{theorem}

\section{Empirical Validation}

\begin{table}[h]
\centering
\begin{tabular}{lcc}
\toprule
\textbf{Source} & \textbf{Compression} & \textbf{Type} \\
\midrule
Cryptographic keys & -0.04\% & Bit-level \\
Audio PCM & 14.75\% & Physics-level \\
Temperature sensor & 35.56\% & Physics-level \\
Accelerometer & 36.67\% & Physics-level \\
Light sensor & 22.65\% & Physics-level \\
Gaussian noise & 3.70\% & Slightly bounded \\
\bottomrule
\end{tabular}
\caption{Compression test results (t-test $p < 0.001$)}
\end{table}

\section{Classification of 40 Distributions}

\begin{table}[h]
\centering
\begin{tabular}{lcc}
\toprule
\textbf{Status} & \textbf{Count} & \textbf{Examples} \\
\midrule
ETHER (broken) & 26 & Gaussian, Poisson, $\chi^2$, $t$, Cauchy \\
REAL (correct) & 10 & Binomial, Hypergeometric, Uniform$[a,b]$ \\
NEEDS FIX & 4 & Beta, Dirichlet, von Mises \\
\bottomrule
\end{tabular}
\caption{Distribution classification}
\end{table}

\section{The Problem with Ethers}

\begin{theorem}[Unbounded Artifact]
Any distribution with unbounded support ($x \in (-\infty, \infty)$) is an artifact of continuous mathematics. Physical reality is bounded by $S_{\text{observable}}$.
\end{theorem}

Examples of ether distributions:
\begin{itemize}
\item \textbf{Gaussian}: Infinite tails assign nonzero probability to impossible events
\item \textbf{Poisson}: Unbounded count with continuous rate parameter
\item \textbf{Power Law}: ``Scale-free'' but physical scales exist
\item \textbf{Cauchy}: Infinite variance is physically impossible
\end{itemize}

\section{Bounded Corrections}

For each ether distribution, the correction is:
\begin{enumerate}
\item Truncate at $S_{\text{observable}}$ boundary
\item Discretize continuous parameters
\item Recognize physical characteristic scales
\item Apply polynomial support $O(n^c)$
\end{enumerate}

\begin{theorem}[Corrected Distribution]
For any ``unbounded'' distribution $P(x)$:
$$P_{\text{bounded}}(x) = \begin{cases}
\frac{P(x)}{Z} & \text{if } x \in S_{\text{observable}} \\
0 & \text{otherwise}
\end{cases}$$
where $Z = \int_{S_{\text{observable}}} P(x) dx$ is the normalization.
\end{theorem}

\section{The Key Identity}

$$\log_2(\sqrt{2}) = \frac{1}{2}$$

Physics-level randomness loses exactly $\frac{1}{2}$ bit per symbol to bounded structure, explaining the $\approx 50\%$ compression observed in physical measurements.

\section{Implications for Statistics}

\begin{enumerate}
\item \textbf{Hypothesis testing}: Test ``Is compression 15-92\%?'' not ``Is distribution Gaussian?''
\item \textbf{Generalization bounds}: Use $S_{\text{observable}}$ size, not $S_{\text{complete}}$
\item \textbf{Parameter estimation}: Discretize at physical resolution
\item \textbf{Inference}: Bounded posteriors converge in polynomial time
\end{enumerate}

\end{document}
