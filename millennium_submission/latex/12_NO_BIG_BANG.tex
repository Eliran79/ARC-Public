\documentclass[11pt]{article}
\usepackage{amsmath,amssymb,amsthm}
\usepackage[margin=1in]{geometry}
\usepackage{booktabs}

\newtheorem{theorem}{Theorem}
\newtheorem{definition}{Definition}
\newtheorem{corollary}{Corollary}

\title{No Big Bang: The Big Bounce\\
\large Singularity Dissolution via Bounded Transformation}
\author{Eliran Sabag}
\date{February 5, 2026}

\begin{document}
\maketitle

\begin{abstract}
We apply the Sabag Bounded Transformation Principle to cosmology, dissolving the Big Bang singularity. The time reversal argument shows: if event horizons preserve information (Two Randomness Theorem), then Big Bang cannot create information from nothing. The result is Big Bounce cosmology with $a_{\min} = 10.0$ at the Planck scale. This eliminates dark energy, inflation, and the horizon problem.
\end{abstract}

\section{The Time Reversal Argument}

\begin{theorem}[Information Preservation]
Event horizons preserve information (proved by Two Randomness Theorem---physical measurements compress 15-92\%, showing bounded structure, not destruction).
\end{theorem}

\begin{theorem}[Time Reversal]
The Big Bang is the time reversal of an event horizon. If information cannot be \textbf{destroyed} at a future singularity, it cannot be \textbf{created} at a past singularity.
\end{theorem}

\begin{corollary}
There is no true $t = 0$ singularity. A pre-Bang state must exist.
\end{corollary}

\section{Big Bounce Cosmology}

\begin{definition}[Bounded Scale Factor]
The Friedmann equation with quantum gravity correction:
$$\left(\frac{\dot{a}}{a}\right)^2 = \frac{8\pi G}{3}\rho\left(1 - \frac{\rho}{\rho_{\text{Planck}}}\right)$$
\end{definition}

At $\rho \to \rho_{\text{Planck}}$, the repulsive term prevents collapse.

\begin{theorem}[Minimum Scale]
$$a(t) \geq a_{\min} = 10.0 \text{ (Planck units)}$$
The universe never reaches zero volume. Information is conserved.
\end{theorem}

\section{What Gets Eliminated}

\begin{table}[h]
\centering
\begin{tabular}{lll}
\toprule
\textbf{Classical} & \textbf{Problem} & \textbf{Bounded Resolution} \\
\midrule
Big Bang & Creation ex nihilo & Big Bounce (continuous) \\
Dark energy & Accelerating expansion & Redshift artifact \\
Inflation & Horizon problem & Not needed (infinite universe) \\
Singularity & Infinite density & Bounded at Planck scale \\
\bottomrule
\end{tabular}
\caption{Classical cosmology problems dissolved}
\end{table}

\section{Redshift as Boundary Artifact}

\begin{theorem}[Observable Horizon]
The observer is confined to $S_{\text{observable}}$ with radius:
$$r_{\text{obs}} = c \times n_{\text{steps}}$$
growing linearly with time, not exponentially.
\end{theorem}

\begin{theorem}[Redshift Reinterpretation]
Cosmological redshift is an $S_{\text{observable}}$ boundary effect (observer confinement), not physical expansion of space.
\end{theorem}

This eliminates:
\begin{itemize}
\item \textbf{Dark energy}: ``Acceleration'' is misinterpreted redshift
\item \textbf{Expansion}: Universe is static and infinite
\item \textbf{Age problem}: No finite age (eternal universe)
\end{itemize}

\section{Einstein Vindicated, Hawking Refuted}

\begin{theorem}[Einstein's Intuition]
``God does not play dice''---quantum measurements have bounded deterministic structure (compress 15-92\%, not 0\%).
\end{theorem}

\begin{theorem}[Hawking Refutation]
Event horizons do not destroy information. The Big Bang did not happen.
\end{theorem}

\section{Falsifiable Predictions}

\begin{enumerate}
\item \textbf{CMB fluctuations}: Should compress 15-92\% (bounded structure from pre-existing state, not quantum creation)
\item \textbf{Redshift measurements}: Should compress 15-92\% (boundary artifact has structure)
\item \textbf{Radioactive decay}: Retrospective compression should detect $>0\%$ (deterministic sequence)
\end{enumerate}

\section{The Key Identity}

$$\log_2(\sqrt{2}) = \frac{1}{2}$$

The discrete-continuous boundary appears at the Planck scale where:
\begin{itemize}
\item Quantum effects dominate ($\hbar$ significant)
\item Gravity is strong ($G$ significant)
\item Classical singularities are artifacts
\end{itemize}

\end{document}
