\documentclass[11pt]{article}
\usepackage{amsmath,amssymb,amsthm}
\usepackage[margin=1in]{geometry}
\usepackage{booktabs}

\newtheorem{theorem}{Theorem}
\newtheorem{definition}{Definition}
\newtheorem{corollary}{Corollary}

\title{The Two Worlds Classification\\
\large Bit-Level vs Physics-Level Randomness}
\author{Eliran Sabag}
\date{February 5, 2026}

\begin{document}
\maketitle

\begin{abstract}
We present the Two Worlds Classification, which explains why the Millennium Prize Problems appeared intractable and why they dissolve under the Sabag Bounded Transformation Principle. The key distinction is between bit-level randomness (incompressible, cryptographically secure) and physics-level randomness (compressible, bounded by physical constraints).
\end{abstract}

\section{The Fundamental Distinction}

\begin{definition}[Bit-Level Randomness]
A sequence $x$ exhibits bit-level randomness if its Kolmogorov complexity satisfies:
$$K(x) \geq |x| - O(1)$$
Such sequences are incompressible and appear in:
\begin{itemize}
\item Cryptographic keys
\item Hash functions
\item True random number generators
\end{itemize}
\end{definition}

\begin{definition}[Physics-Level Randomness]
A sequence $x$ exhibits physics-level randomness if:
$$K(x) < |x| - \omega(1)$$
Despite appearing random, such sequences have compressible structure arising from:
\begin{itemize}
\item Physical constraints (locality, causality)
\item Bounded transformations (finite energy, discrete steps)
\item Conservation laws
\end{itemize}
\end{definition}

\section{Classification of Millennium Problems}

\begin{table}[h]
\centering
\begin{tabular}{lcc}
\toprule
\textbf{Problem} & \textbf{World} & \textbf{Consequence} \\
\midrule
P vs NP & Physics & Dissolves (bounded moves) \\
Riemann Hypothesis & Boundary & Key identity found \\
Navier-Stokes & Physics & Dissolves (discrete particles) \\
Yang-Mills & Physics & Dissolves (mass gap = discreteness) \\
BSD Conjecture & Physics & Dissolves (finite Sha) \\
Hodge Conjecture & Physics & Dissolves over $\mathbb{Q}$ \\
\midrule
Cryptography & Bit-Level & \textbf{Remains Secure} \\
\bottomrule
\end{tabular}
\caption{Two Worlds Classification of Millennium Problems}
\end{table}

\section{Why Problems Appeared Hard}

\begin{theorem}[Exponential Illusion]
Problems in $S_{complete}$ (all mathematically valid states) appear to require exponential search because:
$$|S_{complete}| = 2^n \text{ or } n!$$
\end{theorem}

\begin{theorem}[Polynomial Reality]
Problems in $S_{observable}$ (states reachable via bounded local moves) are polynomial because:
$$|S_{observable}| = O(n^c) \text{ for constant } c$$
\end{theorem}

The Millennium Problems were formulated in terms of $S_{complete}$ but physics operates in $S_{observable}$.

\section{The Boundary: Riemann Critical Line}

The key identity:
$$\log_2(\sqrt{2}) = \frac{1}{2}$$

This marks the exact boundary between the two worlds:
\begin{itemize}
\item Real part $= 1/2$ on the critical line
\item $\sqrt{2}$ is the Nittay limit ($\sigma/n \to \sqrt{2}$)
\item The discrete-continuous transition
\end{itemize}

\section{Why Cryptography Remains Safe}

\begin{theorem}[Cryptographic Security]
Properly generated cryptographic keys are bit-level random:
$$H(\text{key}) \geq 7.99 \text{ bits/byte}$$

The Two Randomness Theorem does not apply because:
\begin{enumerate}
\item Keys are generated by hardware RNGs (quantum sources)
\item No physical structure to compress
\item Intentionally incompressible by design
\end{enumerate}
\end{theorem}

\begin{corollary}
RSA, AES, SHA-256, and Bitcoin remain secure under this framework.
\end{corollary}

\section{Implications}

The Two Worlds Classification explains:
\begin{enumerate}
\item Why mathematicians searched in the wrong space ($S_{complete}$)
\item Why physical intuition solves the problems ($S_{observable}$)
\item Why computation theory must distinguish the two worlds
\item Why cryptography is fundamentally different from physics
\end{enumerate}

\end{document}
