\documentclass[11pt,a4paper]{article}

% Packages
\usepackage[utf8]{inputenc}
\usepackage[T1]{fontenc}
\usepackage{amsmath,amssymb,amsthm}
\usepackage{mathtools}
\usepackage{hyperref}
\usepackage{cleveref}
\usepackage{graphicx}
\usepackage{booktabs}
\usepackage{enumitem}
\usepackage[margin=1in]{geometry}
\usepackage{xcolor}

% Theorem environments
\newtheorem{theorem}{Theorem}[section]
\newtheorem{lemma}[theorem]{Lemma}
\newtheorem{proposition}[theorem]{Proposition}
\newtheorem{corollary}[theorem]{Corollary}
\newtheorem{definition}[theorem]{Definition}
\newtheorem{remark}[theorem]{Remark}
\newtheorem{example}[theorem]{Example}

% Custom commands
\newcommand{\Sobs}{S_{\text{observable}}}
\newcommand{\Scomp}{S_{\text{complete}}}
\newcommand{\bigO}{\mathcal{O}}
\newcommand{\Nittay}{\sqrt{2}}
\newcommand{\half}{\frac{1}{2}}

% Title setup - to be customized per paper
\title{[TITLE]}
\author{Eliran Sabag\\
  \textit{Independent Researcher}\\
  Rishon LeZion, Israel\\
  \texttt{eliran.sbg@gmail.com}
}
\date{February 2026}

\begin{document}

\maketitle

\begin{abstract}
[ABSTRACT]
\end{abstract}

% Keywords
\noindent\textbf{Keywords:} [KEYWORDS]

% MSC Classification
\noindent\textbf{MSC 2020:} [MSC CODES]

\section{Introduction}
\label{sec:intro}

[INTRODUCTION]

\subsection{The Bounded Transformation Principle}

The resolution of this problem relies on a fundamental distinction:

\begin{definition}[Observable vs Complete State Space]
For any computational or mathematical structure:
\begin{align}
\Scomp &= \text{All syntactically valid states} = \bigO(2^n) \\
\Sobs &= \text{States reachable via bounded local moves} = \bigO(n^c)
\end{align}
where $c$ is a constant determined by the move bound.
\end{definition}

\begin{theorem}[Sabag Bounded Transformation Principle]
Physical and computational processes operate exclusively in $\Sobs$, not $\Scomp$.
\end{theorem}

\subsection{The Key Identity}

\begin{theorem}[Nittay-Riemann Connection]
\begin{equation}
\log_2(\Nittay) = \half \quad \text{(exactly)}
\end{equation}
This identity connects the discrete-continuous boundary ($\Nittay$) to the critical line ($\Re(s) = \half$).
\end{theorem}

\section{Preliminaries}
\label{sec:prelim}

[DEFINITIONS AND BACKGROUND]

\section{Main Result}
\label{sec:main}

[MAIN THEOREM AND PROOF]

\section{Verification}
\label{sec:verify}

The claims in this paper are verified by computational experiments. The verification code is available at:

\begin{center}
\texttt{github.com/Eliran79/ARC-Public}
\end{center}

[VERIFICATION DETAILS]

\section{Conclusion}
\label{sec:conclusion}

[CONCLUSION]

\section*{Acknowledgments}

This work was developed with assistance from Claude (Anthropic).

\bibliographystyle{plain}
\begin{thebibliography}{9}

\bibitem{sabag2026arc}
E. Sabag.
\newblock ARC: The Sabag Bounded Transformation Principle.
\newblock GitHub, 2026.
\newblock \url{https://github.com/Eliran79/ARC-Public}

\bibitem{clay}
Clay Mathematics Institute.
\newblock Millennium Prize Problems.
\newblock \url{https://www.claymath.org/millennium-problems}

\end{thebibliography}

\appendix

\section{Proof Details}
\label{app:proofs}

[DETAILED PROOFS]

\section{Verification Code}
\label{app:code}

[CODE LISTINGS]

\end{document}
